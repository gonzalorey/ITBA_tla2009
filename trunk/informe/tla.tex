\documentclass[a4paper,11pt]{article}
\usepackage[latin1]{inputenc}
\usepackage[spanish]{babel}
\usepackage{amsmath}
\usepackage{amsfonts}
\usepackage{amssymb}
\usepackage{graphicx}
\usepackage[usenames]{color}
\usepackage{gastex}


\title{Trabajo Pr\'actico \\ Teor\'ia de Lenguaje, Aut\'omatas y Compiladores }
\author{Pablo Mui\~na, Augusto Mc Intosh, Marcos Pianelli}


\begin{document}

\maketitle

\section{ \'Indice }

\begin{enumerate}
\item Introducci\'on
\item Consideraciones realizadas
\item Wikitext : Tokens aceptados y sus respectivos aut\'omatas
\item Modo de uso de los programas
\end{enumerate}


\section{ Introducci\'on }



\section{ Consideraciones realizadas }

Se tuvo en cuenta que los tokens que el programa aceptar\'ia solo ser\'ian las referenciadas en el \emph{cheetsheet} provisto en el material de referencia.

\section{Wikitext : Tokens aceptados y sus respectivos aut\'omatas}

Los tokens a detectar seg\'un el \emph{cheetsheet} son los siguientes :\\


\begin{center}
 %\begin{verbatim}
 \begin{enumerate}
\item\begin{verbatim}[[\end{verbatim} 
\item	\begin{verbatim}|\end{verbatim} 
\item	\begin{verbatim}]]\end{verbatim} 
\item	\begin{verbatim}[\end{verbatim} 
\item	\begin{verbatim} \n\end{verbatim} 
\item	\begin{verbatim}]\end{verbatim} 
\item	\begin{verbatim}http://\end{verbatim} 
\item	\begin{verbatim}[http://\end{verbatim} 
\item	\begin{verbatim}''\end{verbatim} 
\item	\begin{verbatim}'''\end{verbatim} 
\item	\begin{verbatim}'''''\end{verbatim} 

\item	\begin{verbatim}#REDIRECT\end{verbatim} 
\item	\begin{verbatim}==\end{verbatim} 
\item	\begin{verbatim}===\end{verbatim} 
\item	\begin{verbatim}====\end{verbatim} 
\item	\begin{verbatim}=====\end{verbatim} 
\item	\begin{verbatim}======\end{verbatim} 
\item	\begin{verbatim}<ref name="\end{verbatim} 
\item	\begin{verbatim}">\end{verbatim} 
\item	\begin{verbatim}" />\end{verbatim} 
\item	\begin{verbatim},\end{verbatim} 
\item	\begin{verbatim}''\end{verbatim} 
\item	\begin{verbatim}'''\end{verbatim} 
\item	\begin{verbatim}*\end{verbatim} 
\item	\begin{verbatim}#\end{verbatim} 
\item	\begin{verbatim}:\end{verbatim} 
\item	\begin{verbatim}~~~\end{verbatim} 
\item	\begin{verbatim}~~~~\end{verbatim} 
\item	\begin{verbatim}~~~~~\end{verbatim} 
\item	\begin{verbatim}[[File:\end{verbatim}
\end{enumerate}
 %\end{verbatim} 
\end{center}

De la cuales se obtuvieron las siguientes expresiones regulares y Aut\'omatas:
\\\\

\quad 1.
\begin{center}
%\begin{picture}(width desde margen derecho,height desde margen superior)(x-off,y-off)
%470-85 = 300
\setlength{\unitlength}{1pt}
\begin{picture}(300,60)(0,0)
	\node[Nmarks={i}](A)(0,15){$q_0$}
	\node(B)(50,15){$q_1$}
	\node[Nmarks={r}](C)(100,15){$q_2$}
	\drawedge[curvedepth=6](A,B){[}
	\drawedge[curvedepth=6](B,C){[}
\end{picture}
\end{center}


\quad 2.
\begin{center}
%\begin{picture}(width desde margen derecho,height desde margen superior)(x-off,y-off)
%470-85 = 300
\setlength{\unitlength}{1pt}
\begin{picture}(300,60)(0,0)
	\node[Nmarks={i}](A)(0,15){$q_0$}
	\node[Nmarks={r}](B)(50,15){$q_2$}
	\drawedge[curvedepth=6](A,B){|}
\end{picture}
\end{center}


\quad 16.
\begin{center}
%\begin{picture}(width desde margen derecho,height desde margen superior)(x-off,y-off)
%470-85 = 300
\setlength{\unitlength}{1pt}
\begin{picture}(300,60)(0,0)
	\node[Nmarks={i}](A)(0,15){$q_0$}
	\node(B)(50,15){$q_1$}
	\node(C)(100,15){$q_2$}
	\node(D)(150,15){$q_3$}
	\node(E)(200,15){$q_4$}
	\node[Nmarks={r}](F)(250,15){$q_5$}
	\drawedge[curvedepth=6](A,B){=}
	\drawedge[curvedepth=6](B,C){=}
	\drawedge[curvedepth=6](C,D){=}
	\drawedge[curvedepth=6](D,E){=}
	\drawedge[curvedepth=6](E,F){=}
\end{picture}
\end{center}


\quad 17.
\begin{center}
%\begin{picture}(width desde margen derecho,height desde margen superior)(x-off,y-off)
%470-85 = 300
\setlength{\unitlength}{1pt}
\begin{picture}(300,60)(0,0)
	\node[Nmarks={i}](A)(0,15){$q_0$}
	\node(B)(50,15){$q_1$}
	\node(C)(100,15){$q_2$}
	\node(D)(150,15){$q_3$}
	\node(E)(200,15){$q_4$}
	\node(F)(250,15){$q_5$}
	\node[Nmarks={r}](G)(300,15){$q_6$}
	\drawedge[curvedepth=6](A,B){=}
	\drawedge[curvedepth=6](B,C){=}
	\drawedge[curvedepth=6](C,D){=}
	\drawedge[curvedepth=6](D,E){=}
	\drawedge[curvedepth=6](E,F){=}
	\drawedge[curvedepth=6](F,G){=}
\end{picture}
\end{center}



.... DO MORE






\section {Modo de uso de los programas}

Para poder utilizar el programa se tiene que ejecutar el MakeFile, posici\'onandose en la carpeta \emph{tla2009} y ejecutando el comando \emph{make} en el shell de preferencia.\\
\begin{center}
 \begin{verbatim}
  Make
 \end{verbatim}
\end{center}

Luego para ejecutar el programa de conversi\'on de WikiText a HTML se tipea : \\

\begin{center}
 \begin{verbatim}
  ./pwiki inputFile outputFile 
 \end{verbatim}
\end{center}

Donde inputFile es el archivo de entrada el cual alberga el texto de formato WikiText y outputFile es el archivo de salida con formato HTML.\\



\end{document}

